\section{Risico’s}
Het ontwerpen van een windturbinepark en het opstellen van een onderhoudsplan zijn complexe projecten waarbij verschillende interne en externe risico's kunnen optreden. De risico´s kunnen verschillen van tijdsgebonden problemen tot technische complicaties. 

Het niet volledig begrijpen van de doelstellingen en eisen van het project, en/of onvoldoende communicatie tussen teamleden, belanghebbenden en experts kan resulteren in misverstanden, vertragingen, onsamenhangende of niet relevante informatie en overige fouten in de documenten. Daarom is, om dit te voorkomen, vooraf het plan van aanpak opgesteld. Daarnaast zal aan het begin van het project een analysefase plaatsvinden waarin de doelstellingen duidelijk worden voor de projectleden. Verder wordt er iedere week gecommuniceerd tussen de projectleden en worden wijzigingen in Github genoteerd. Zo wordt ervoor gezorgd dat op dit front niks fout gaat. 

Gebrek aan kennis kan ook zorgen voor problemen. Het projectteam heeft mogelijk onvoldoende kennis of ervaring met windturbineparkontwerp en -onderhoud, wat kan leiden tot onjuiste aanbevelingen. Het is daarom van belang dat het projectteam genoeg onderzoek doet en hun kennis uitbreid gedurende het project. Als het niet lukt de benodigde informatie te vinden of kennis op te doen, zal er naar alternatieve externen moeten worden gezocht voor hulp. Dit zal ervoor zorgen dat gebrek aan kennis geen barricade zal vormen, de teamleden hun kennis uitbreiden en toepassen om zo toch correcte en goede kwaliteit producten te kunnen leveren. 

Tijdsgebrek is ook een risico. Tijdsdruk om het parkontwerp en onderhoudsplan binnen strakke deadlines op te stellen, kan leiden tot gehaaste beslissingen en onvolledigheid van de documenten. Door goed te plannen en voor te bereiden zal dit worden voorkomen. 

Technologische veranderingen kunnen goed zijn, echter kan het ook voor problemen zorgen. Snelle technologische vooruitgang kan van invloed zijn op de keuze van apparatuur en systemen in het ontwerp en het onderhoudsplan, wat herziening noodzakelijk kan maken. Dit is een risico dat niet voorkomen kan worden. Het effect kan echter wel beperkt worden door op een bepaald moment een definitief besluit te nemen over de apparatuur en systemen die gebruikt zullen worden. 

Ten slotte zijn er nog financiële risico's. Beperkt budget kan de mogelijkheden limiteren. Daarnaast kunnen verschillende onverwachte gebeurtenissen gedurende het project zorgen voor verhoogde kosten. Mocht dit voorkomen, dan zal de situatie met de opdrachtgever besproken worden en zal gezocht worden naar een oplossing.
\section{Wat zijn de te verwachte stromen door de kabels en de bijbehorende belangrijke gegevens?}

De stromen die door de kabels zullen stromen hangt af van meerdere factoren. Ten eerste is het aantal turbines van belang. In het windpark zijn, zoals in het vorige hoofdstuk aangegeven, de turbines met groepen van vier, drie of één via een inter array kabel verbonden met het TenneT-station. Het totale werkelijk vermogen P wat dan door de kabel zal moeten verschilt dus van 72MW tot 18MW. Om de stromen te kunnen berekenen zal voor de volgende berekening worden aangenomen dat het totale werkelijk vermogen P overeen komt met het totale schijnbaar vermogen S. S zal voor de verschillende aantallen turbines dus drie waardes hebben. 
De verwachte stromen per fase kabel zijn te berekenen door middel van formule \ref{eq:31}. 

\begin{equation} \label{eq:31}
I\textsubscript{fase} = \frac{S/3}{U/ \sqrt{3}} = \frac{72.000.000/3}{66.000/ \sqrt{3}} = 629,84 A
\end{equation}



\section{Aan welke onderdelen moet onderhoud worden uitgevoerd en wat voor soort onderhoud betreft dit?}

Binnen het domein van onderhoud aan offshore windturbines zijn verschillende componenten onderhevig aan inspectie, reparatie, en indien nodig, vervanging. Bij klein onderhoud worden veelal kleine, maar essentiële onderdelen aangepakt. Dit omvat bijvoorbeeld het vervangen van kleine componenten die geen jack-up vessel vereisen. Dergelijke componenten kunnen betrekking hebben op elektrische bedrading, sensoren, beveiligingssystemen en andere delen van de turbine die cruciaal zijn voor de goede werking ervan \cite{Eneco_presentatie}.

Bladreparaties vormen een ander aspect van klein onderhoud. De bladen van een windturbine worden blootgesteld aan aanzienlijke krachten en omstandigheden, wat kan leiden tot schade aan het oppervlak (gel coat) en zelfs de structuur van glasvezel. Deze schade kan optreden als gevolg van slijtage door weersomstandigheden, vogelaanvaringen of andere externe factoren. Kleine reparaties aan de bladen omvatten het herstellen van deze schade, wat kan variëren van het aanbrengen van beschermende coatings tot het repareren van kleine scheuren of deuken \cite{Eneco_presentatie}.

Groot onderhoud omvat ingrijpende operaties waarbij componenten vaak volledig worden vervangen. Een voorbeeld hiervan is de vervanging van de tandwielkast, een cruciaal onderdeel voor het omzetten van de langzame rotatie van de turbinebladen naar de snellere rotatie die nodig is om elektriciteit te genereren. Ook de generator kan het onderwerp zijn van vervanging tijdens grootschalig onderhoud \cite{Eneco_presentatie}.

Inspectie en reparatie van kabels zijn van groot belang, omdat beschadigde kabels kunnen leiden tot productieverlies en aanzienlijke kosten. Schade aan de kabels kan optreden door contact met scheepvaart, ankers die over de zeebodem slepen, of door vortex induced vibrations (VIV) als gevolg van onstabiele zeebodemomstandigheden. Herstelwerkzaamheden kunnen variëren van het herstellen van de isolatielaag tot het vervangen van volledige secties van de kabel \cite{Eneco_presentatie}.
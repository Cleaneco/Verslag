\section{De producten}
\subsection{Plan van Aanpak}
Het eerste product dat geleverd zal worden is het plan van aanpak. In het plan van aanpak staan alle plannen en belangrijke informatie voor het opzetten van het project. Dit document zal worden geleverd voordat aan het project begonnen wordt. Daarnaast zal het ook gepresenteerd worden in een korte pitch van twee minuten. Het doel van het plan van aanpak is, het creëren van duidelijkheid over: de verwachtingen van de opdrachtgever, de doelen waar naartoe moet worden gewerkt en hoe, en de belangrijke informatie over het project, zoals mogelijke risico's, te ondernemen acties en de werkwijze. 

\subsection{Parkontwerp}
Het tweede product dat gemaakt en geleverd zal moeten worden is het parkontwerp. Het windturbinepark staat in dit project centraal, hiervan zal dan ook een uitgewerkt ontwerp worden geleverd. Hierin worden technische aspecten behandeld, zo zullen twee type windturbines worden uitgewerkt, voor beide zal ook de optimale positionering worden bepaald voor de beste opbrengst. Dit zal bepaald worden door het gedrag van de wind op de kavels te analyseren. Er zal ook worden ingegaan op de verwachte energieopbrengsten voor de gekozen turbines. 

Bij het parkontwerp hoort natuurlijk ook een uitwerking van de bekabeling, de specificaties van de kabels zelf, en de uitwerking van de onderlinge connecties en die met het hoogspanningsstation. Bij het ontwerpen van het windturbinepark zal ook rekening worden gehouden met het effect dat alle hiervoor genoemde elementen zullen hebben op de onderhoud van het park. 

\subsection{Onderhoudsplan}
Het derde product dat geleverd zal worden is het onderhoudsplan. In het onderhoudsplan zal komen welke componenten, met welke frequentie, onderhoud nodig zullen hebben. Ook wordt ingegaan op de soort onderhoud en hoe dit moet gebeuren. De frequentie zal worden gebaseerd op en onderbouwd door statistische gegevens over de verschillende componenten van windturbineparken. Met de frequentie zal vervolgens een onderhoudsstrategie gemaakt worden dat ook verwerkt wordt in het onderhoudsplan. 

Verder zal er in het onderhoudsplan worden besproken hoe de conditie van de componenten bewaakt zal worden. Ten slotte zullen de risico's van het plegen van onderhoud worden uitgewerkt in het plan. 

\subsection{Adviesrapport}
Als laatste product zal het adviesrapport worden opgesteld voor Worley. In het adviesrapport zullen de belangrijkste aspecten van de vorige producten worden samengevat. Zo zal de probleemstelling van het project aan bod komen samen met de gekozen turbine types en verwachte energieopbrengsten. Verder zal het parkontwerp inclusief bekabeling en de bijbehorende kabelberekeningen in het adviesrapport worden verwerkt. Ten slotte zal ook het onderhoudsplan niet uit het adviesrapport worden weggelaten. Ter afsluiting van het project zal dit adviesrapport (aan Worley) worden gepresenteerd. 

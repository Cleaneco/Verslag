\section{Kosten en Opbrengsten}
    \subsection{Financiële kosten en budget}
    Bouwkosten Windpark: Hoewel specifieke bouwkosten niet worden vermeld in de verstrekte informatie, kan worden aangenomen dat de bouw van een windpark van 756 MW met 54 windturbines aanzienlijke kapitaalinvesteringen met zich meebrengt. Dit omvat de kosten voor het ontwerpen, fabriceren en installeren van de windturbines en de benodigde infrastructuur op zee.

    Milieueffectrapportages en Locatiestudies: Er wordt vermeld dat Ecowende kosten draagt voor milieueffectrapportages en locatiestudies. Deze studies zijn nodig om de impact van het windpark op het milieu en de omgeving te beoordelen en te minimaliseren.

    Overheidsinkomsten: De overheid ontvangt € 63,5 miljoen van de partijen als gevolg van financiële biedingen en bijdragen van Ecowende. Dit geld wordt gebruikt om nieuwe windparken beter te integreren in de omgeving en voor overige activiteiten op de Noordzee.

    % \subsection{Verklarende woordenlijst}
Tijdslijn van het Project:

    2022: Ecowende, een \gls{joint venture} van Shell en Eneco, wint de vergunning voor het windpark Hollandse Kust (west) kavel VI.

    2023: Ontwerp en plannen worden gemaakt, en de bouw van het windpark is gepland om te beginnen in dit jaar.

    2026: Het windpark wordt naar verwachting volledig operationeel en levert elektriciteit.

Het maken van het windturbineparkontwerp en het onderhoudsplan zal gebeuren in een periode van 17 weken. Hiervan zal aan het begin ongeveer 60\% van de tijd besteed worden aan het maken van het windparkontwerp, de overige tijd zal worden gestoken in het realiseren van het onderhoudsplan en het verwerken van beide producten in het adviesrapport.\cite{studiewijzer} 

\subsection{Opbrengsten}

    Duurzame Energievoorziening: Het windpark Hollandse Kust (west) zal naar verwachting een vermogen van 756 MW genereren. Dit komt overeen met ongeveer 3\% van de Nederlandse elektriciteitsvraag, wat voldoende is om ongeveer een miljoen huishoudens van elektriciteit te voorzien. De opbrengst voor eindgebruikers is een toename van duurzame energievoorziening en een verminderde afhankelijkheid van niet-hernieuwbare bronnen zoals fossiele brandstoffen.

    Milieubescherming: Het ontwerp van het windpark is '\gls{natuurinclusief}' en omvat maatregelen om de impact op de natuur en het milieu te minimaliseren. Dit omvat bijvoorbeeld het creëren van een gezond ecosysteem, het minimaliseren van de impact op vogels, het onderwaterleven, en het stimuleren van biodiversiteit onder water. De opbrengst hier is een stap in de richting van milieubescherming en behoud van biodiversiteit.

    Toekomstige Groene Energie: Het windpark maakt deel uit van de groeiende inspanningen om groene energie in Nederland te vergroten. Dit draagt bij aan de ambitie om tegen 2030 ongeveer 21 GW aan windenergie op zee te hebben, wat een duurzame energiebron voor de toekomst zal opleveren.

    Economische Impuls: De bouw van windparken op zee kan een economische impuls geven aan de regio, met potentiële werkgelegenheidskansen in havens zoals Vlissingen, IJmuiden, Den Helder en de Eemshaven.

Samengevat, het project levert duurzame energie, milieubescherming, toekomstige groene energiebronnen en economische kansen op voor de eindgebruikers en de samenleving als geheel.\cite{Windenergiegebied}\cite{ShellEneco}\cite{Windopzee}

% % Opbrengst:

% % https://www.noordzeeloket.nl/functies-gebruik/windenergie/doorvaart-medegebruik/hollandse-kust-west/

% % https://www.rvo.nl/onderwerpen/bureau-energieprojecten/afgesloten-projecten/woz-hollandse-kust-west-kavels-vi-vii

% % https://www.rijksoverheid.nl/actueel/nieuws/2022/12/15/shell-en-eneco-winnen-tender-windpark-op-zee-hollandse-kust-west
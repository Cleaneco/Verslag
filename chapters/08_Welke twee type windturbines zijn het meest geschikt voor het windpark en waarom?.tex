% \section{Welke twee type windturbines zijn het meest geschikt voor het windpark en waarom?}
% Er zullen meerdere maatregelen worden genomen om de kwaliteit van de producten te waarborgen. Voor alle producten geldt dat er een eindredactie zal plaatsvinden. Bij de eindredactie zullen de documentaties worden gecontroleerd op inhoudelijke en verslagtechnische correctheid. Ook zullen voor alle producten alleen betrouwbare bronnen geraadpleegd worden met relevante informatie en actuele autoriteit. Daarnaast zal verwezen worden naar alle bronnen volgens de verslagtechnische eisen, met correcte toepassing van IEEE-richtlijnen. Verder zullen, ter verbetering van de leesbaarheid van het document, alle moeilijke termen genoteerd worden in een begrippenlijst.

% \subsection{Planning en taakverdelingen}
% Naast een eindredactie wordt, om de kwaliteit van de producten en de timing van de levering hiervan te garanderen, voor een goede organisatie gezorgd. Hieronder vallen een goede taakverdeling, planning en communicatie. Voor de taakverdeling en planning wordt het platform Github gebruikt, hierop zijn de taken per projectlid ingedeeld en ingepland. Zo kan worden verzekerd dat iedereen zich aan alle deadlines houdt. Als dit niet het geval blijkt te zijn, wordt dit besproken en zijn er consequenties. Naast Github is de planning ook in grote lijnen uitgewerkt in het plan van aanpak. 

% \subsection{Onderzoeksvragen}
% Met als doel het waarborgen van de structuur en kwaliteit van de documentaties, zal de probleemstelling worden opgedeeld in kleinere problemen. Deze problemen zullen worden uitgewerkt met behulp van onderzoeksvragen.\cite{projecthandleiding}
% De hoofdvraag voor het gehele project is: "\textbf{Hoe kunnen de eisen volgens het energieakkoord \cite{energieakkoord} van 2013 behaald worden met behulp van een windturbinepark op kavel VI en VII?}". De hoofdvraag zal worden beantwoord door middel van meerdere deelvragen. Voor tussenproduct 1, het park ontwerp, zullen de volgende vragen beantwoord worden\cite{projecthandleiding}:
% \begin{itemize}
%     \item Welke twee type windturbines zijn het meest geschikt voor het windpark en waarom?
%     \item Wat is de beste positionering van de windturbines voor optimale opbrengst, efficiënte bekabeling en makkelijker onderhoud?  
%     \item Welke kabels zijn geschikt voor het transporteren van de energie van turbines tot het hoogspanningsstation met de verwachte stromen?
%     \item Hoe ziet de bekabeling van het windpark eruit en waarom?
%     \item Wat  de beste soort spanning voor dit windturbinepark is, gelijk- of wisselspanning en welke effecten heeft dit op het park? %Bespreek transformator boxen
%     \item Welk invloed hebben de gemaakte keuzes op de onderhoud van het park? 
% \end{itemize}

% Ter beantwoording van tussenproduct 2, het onderhoudsplan, zijn de volgende deelvragen opgesteld:
% \begin{itemize}
%     \item Hoe wordt de conditie van de componenten waaruit het windturbinepark bestaat bewaakt? 
%     \item Welk onderhoud moet er plaatsvinden aan het windturbinepark?
%     \item Hoe frequent moet er onderhoud plaatsvinden en hoe is dit bepaald?
%     \item Hoe ziet de onderhoudsstrategie eruit? 
%     \item Welke materiële en financiële risico's zijn er bij het plegen van onderhoud?
% \end{itemize}

% Deze vragen uitgebreid beantwoorden garandeert dat alle belangrijke aspecten voor dit project behandeld worden. 

\section{Turbines}
\subsection{Introductie}
In de huidige wereld van hernieuwbare energie staan offshore windturbines als pion van een groenere toekomst. De keuze voor de juiste windturbine voor een project is cruciaal, en het verkrijgen van nauwkeurige en betrouwbare informatie over deze turbines is van het groot belang. Onderzoek heeft getoond dat er verscheidene turbines en fabrikanten zijn, waaronder de GE Haliade-X 14 MW, de Goldwind GWH252-16MW, de Vestas V236, de NREL 12MW en de NREL 18MW.
\subsection{GE Haliade-X 14 MW}
De GE Haliade-X 14 MW is een krachtige offshore windturbine met een aanzienlijk nominaal vermogen van 14 MW. Met een indrukwekkende rotordiameter van 220 meteren een jaarlijkse energieproductie van ongeveer 74 gigawattuur, speelt deze turbine een significante rol in de offshore windenergiesector. \cite{GEHalideX}\cite{TNOHalideX}\cite{TopsectorEnergieHalideX}\cite{OffshoreWindHalideX}\cite{AandrijftechniekHalideX}\cite{PonderaHalideX}
\subsection{Vestas V236}
De Vestas V236, met een vermogen van 15 MW en een rotordiameter van 236 meter, vertegenwoordigt eveneens een aanzienlijke keuze binnen de offshore windenergie sector. Desondanks wordt hij op het gebied van nominaal vermogen en jaarlijkse energieproductie overtroffen door de MySE-windturbine. \cite{Vestas15MW}\cite{Vestas}\cite{VestasJourney}\cite{WindTurbineModels}\cite{Electrek}\cite{OffshoreWindBiz3}
\subsection{Ming Yang MySE 16-260}
De MySE-windturbine, ontwikkeld door het Chinese bedrijf MingYang, heeft de aandacht getrokken vanwege zijn indrukwekkende technische specificaties en efficiëntie. Met een rotordiameter van 252 meter behoort deze windturbine tot de grootste in zijn categorie, wat hem in staat stelt om krachtige offshore winden effectief te benutten. Met een nominaal vermogen van 16 MW kan de MySE aanzienlijke hoeveelheden elektriciteit genereren, zelfs onder uitdagende omstandigheden.
Een opvallend kenmerk van de MySE-windturbine is zijn vermogen om windbronnen efficiënter te benutten dan sommige concurrenten. Dit wordt ondersteund door gegevens die aantonen dat de MySE vergelijkbare of zelfs grotere hoeveelheden elektriciteit kan produceren dan windturbines met een hoger nominaal vermogen.\cite{NewAtlas}\cite{MySEWebsite}\cite{OffshoreWindMySE}\cite{YouTube}\cite{MySEWebsite2}
\subsection{Goldwind GWH252-16MW}
De Goldwind GWH252-16MW, met een indrukwekkende rotordiameter van 252 meter, toont een opmerkelijke technologische prestatie. Echter, de beschikbare gegevens suggereren dat de MySE-windturbine van MingYang nog steeds vergelijkbare of zelfs grotere hoeveelheden elektriciteit kan genereren, wat impliceert dat de MySE mogelijk efficiënter is in het benutten van windenergiebronnen.\cite{Goldwind}\cite{4COffshore}
\subsection{National Renewable Energy Laboratory}
Na grondige overweging en uitgebreide vergelijkingen met deze concurrenten, is er besloten om de focus te richten op de windturbines van het National Renewable Energy Laboratory (NREL). En dat is om goede redenen.\cite{NRELVision}\cite{NRELAwards}
\subsubsection{NREL12MW}
De NREL 12MW Offshore windturbine, hoewel iets lager in vermogen dan zijn grotere broer, heeft ook indrukwekkende specificaties en prestaties. Met een rotordiameter van 222 meter en een geschatte jaarlijkse energieproductie van 12 gigawattuur, blijft deze windturbine een krachtige en efficiënte optie voor offshore windenergieprojecten.\cite{NRELOregonWindStudy}\cite{NRELATB2020}\cite{NRELReference12MW}\cite{NRELCsv12MW}

Specificaties van de NREL 12W Offshore windturbine:
\begin{itemize}
    \item \textbf{Naam:} NREL 12MW Offshore windturbine
    \item \textbf{Nominaal vermogen:} 12.000 kW
    \item \textbf{Geschatte windsnelheid:} 11 m/s
    \item \textbf{Aanvang windsnelheid:} 3 m/s
    \item \textbf{Stop windsnelheid:} 25 m/s
    \item \textbf{Rotordiameter:} 222 meter
    \item \textbf{Hubhoogte:} 136 meter
\end{itemize}


\subsubsection{NREL18MW}
Specificaties van de NREL 18MW Offshore windturbine:
\begin{itemize}
    \item \textbf{Naam:} NREL 18MW Offshore windturbine
    \item \textbf{Nominaal vermogen:} 18.000 kW
    \item \textbf{Geschatte windsnelheid:} 11 m/s
    \item \textbf{Aanvang windsnelheid:} 3 m/s
    \item \textbf{Stop windsnelheid:} 25 m/s
    \item \textbf{Rotordiameter:} 263 meter
    \item \textbf{Hubhoogte:} 150 meter
\end{itemize}

De NREL 18MW Offshore windturbine biedt enkele opmerkelijke voordelen die hem tot een uitstekende keuze maken. Met een nominaal vermogen van 18 MW, een indrukwekkende rotordiameter van 263 meter en een jaarlijkse energieproductie van maar liefst 80 gigawattuur, is deze windturbine een krachtige oplossing op zee. Zijn hoge capaciteitsfactor zorgt ervoor dat hij consistent een hoog rendement kan leveren en meer elektriciteit aan het net kan leveren.\cite{NRELReference18MW}\cite{NRELCSV18MW}\cite{NRELATB2020Specific}
\subsubsection{Vergelijking}
Bovendien bieden de NREL 18MW en NREL 12MW Offshore windturbines de geruststellende zekerheid van betrouwbare gegevens en informatie die online open source beschikbaar zijn. Om deze reden kan er een nauwkeurige berekeningen gemaakt worden en het project op de meest solide basis te bouwen.\cite{NRELTurbineModels}

De keuze voor de NREL 18MW en NREL 12MW Offshore windturbines is niet alleen gebaseerd op hun indrukwekkende technische specificaties, maar ook op de visie van het National Renewable Energy Laboratory om een schone energietoekomst voor de wereld te creëren. Hun inzet voor energiegerechtigheid en inclusieve ontwerpen sluit naadloos aan bij onze eigen waarden en doelen. \cite{NRELDiversity}\cite{NRELSustainability}\cite{NRELTurbineModels}
\subsection{Conclusie}
Terwijl de windturbines van MingYang, Goldwind en Vestas ongetwijfeld indrukwekkend zijn, hebben de NREL 18MW en NREL 12MW Offshore windturbines de overhand in termen van vermogen, efficiëntie en betrouwbare informatie. Het is een keuze die ons dichter bij een groenere, schonere toekomst brengt, waarin offshore windenergie een cruciale rol speelt in de transitie naar duurzame energiebronnen. Met de NREL 18MW en NREL 12MW Offshore windturbines aan onze zijde zijn we vastbesloten om een positieve impact te hebben op de wereld van hernieuwbare energie en een duurzamere planeet te creëren voor toekomstige generaties.
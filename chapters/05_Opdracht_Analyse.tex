\section{Opdracht analyse}


% \section{Projectgrenzen}
% \subsection{Wat wel}
% \begin{itemize}
%     \item Wij schrijven een plan van aanpak.
%     \item Wij doen onderzoek naar de technische aspecten en de energie opbrengsten.
%     \item Wjj doen onderzoek naar de verschillende windturbines, transformatoren, bekabeling en toebehoren.
%     \item Wij schrijven een onderhoudsplan voor 25 jaar waarbij we gebruik maken van levensduur en statistische gegevens.
%     \item Wij maken een parkontwerp waarbij rekening wordt gehouden met een optimale plaatsing van de windturbines.
%     \item Wij presenteren ons parkontwerp en onderhoudsplan aan onze opdrachtgever: Worley.
% \end{itemize}
% \subsection{Wat niet}
% \begin{itemize}
%     \item Wij vragen geen vergunning aan voor een \gls{offshore} windpark op zee.
%     \item Wij vragen geen subsidie aan bij de Nederlandse overheid.
%     \item Wij beheren geen elektrische infrastructuur (wordt gedaan door TenneT\cite{energieakkoord}).
%     \item Wij realiseren geen \gls{offshore} windpark op zee.
%     \item Wij zijn niet verantwoordelijk voor oplopende problemen tijdens de realisatie fase van het \gls{offshore} windpark op zee.
% \end{itemize}
% \subsection{Verwachte tijdsduur}
% Op dinsdag 29 augustus zijn we begonnen aan het project \textit{Duurzame Energie} en we plannen om hier 17 weken aan te besteden. In week 19 staat onze eindpresentatie gepland voor onze opdrachtgever Worley, die zal plaatsvinden op hun kantoor in Den Haag. 
% We schatten in dat we gedurende dit project ongeveer 16 x 2 contacturen en 16 x 6 zelfstudie-uren zullen besteden\cite{studiewijzer}.


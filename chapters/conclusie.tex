\section{Conclusie}
Aan het begin van het project zijn in het Plan van Aanpak de onderzoeksvragen voor dit rapport opgesteld. Deze zijn gedurende de loop van het project echter aangepast, omdat er meer kennis over de onderwerpen is opgedaan en de nieuwe onderzoeksvragen het doel beter dienen. Het doel van dit rapport is een duidelijk, onderbouwd advies leveren aan de opdrachtgever, waarin een uitgewerkt parkontwerp en onderhoudsplan staan voor Hollandse Kust West (kavel VI en VII). De bijgewerkte onderzoeksvragen zijn volledig uitgewerkt en beantwoord in het verslag. 




%Ingenieurs- en consultancybureau Worley heeft Clean Eco ingeschakeld voor de ontwikkeling van een ontwerp voor een milieuvriendelijk windturbinepark. Om het project met succes af te ronden, zal Clean Eco verschillende essentiële componenten leveren, waaronder twee technische parkontwerpen met diverse configuraties, een onderhoudsplan en een adviserend rapport. Door de onderzoeksvragen te beantwoorden, zal Clean Eco deze onderdelen gedetailleerd uitwerken. Aan het einde van het project zal op basis van het uitgevoerde onderzoek een adviserend rapport worden opgesteld. Dit rapport omvat aanbevelingen voor het optimale parkontwerp, de meest effectieve onderhoudsstrategie en een prognose van de verwachte opbrengsten.
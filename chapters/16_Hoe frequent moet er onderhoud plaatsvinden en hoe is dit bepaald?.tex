\section{Hoe frequent moet er onderhoud plaatsvinden en hoe is dit bepaald?}

De frequentie van onderhoud hangt af van verschillende factoren, waaronder de omgevingsomstandigheden, de levensduur van componenten en de mate van slijtage. Klein onderhoud, dat minder tijdrovend is, kan meerdere keren per jaar plaatsvinden, afhankelijk van de behoeften en staat van de turbines \cite{Eneco_presentatie}.

Groot onderhoud, dat complexer is en grootschaliger planning vereist, wordt minder vaak uitgevoerd, mogelijk op jaarlijkse of tweejaarlijkse basis. De planning van groot onderhoud wordt vaak gebaseerd op de verwachte levensduur van kritieke componenten en de prestaties van de windturbine \cite{Eneco_presentatie}.

Inspecties van bladen en kabels kunnen regelmatig zijn, vaak op jaarlijkse of halfjaarlijkse basis. Deze frequentie kan variëren afhankelijk van factoren zoals de zeeomstandigheden, mate van slijtage en de resultaten van eerdere inspecties. Het doel is om potentiële problemen vroegtijdig te identificeren en tijdig te interveniëren om grotere schade te voorkomen \cite{Eneco_presentatie}.

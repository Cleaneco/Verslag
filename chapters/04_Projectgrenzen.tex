\section{Projectgrenzen}
\subsection{Wat wel}
\begin{itemize}
\item Een plan van aanpak zal worden opgesteld.
\item Er wordt onderzoek uitgevoerd naar aanleiding van de technische aspecten en vereiste energieopbrengsten.
\item Onderzoek wordt verricht naar verschillende windturbines, transformatoren, bekabeling en toebehoren.
\item Een onderhoudsplan voor 25 jaar wordt geschreven, waarbij gebruik wordt gemaakt van levensduur- en statistische gegevens.
\item Er wordt een parkontwerp gemaakt waarbij rekening wordt gehouden met een optimale plaatsing van de windturbines.
\item Het parkontwerp en onderhoudsplan worden gepresenteerd aan de opdrachtgever, Worley.
\end{itemize}
\subsection{Wat niet}
\begin{itemize}
\item Er wordt geen vergunning aangevraagd voor een \gls{offshore} windpark op zee.
\item \gls{subsidie} bij de Nederlandse overheid zal niet worden aangevraagd.
\item Het beheer over de elektrische infrastructuur zal niet door Cleaneco gedaan worden (dit wordt uitgevoerd door TenneT\cite{energieakkoord}).
\item Er wordt geen \gls{offshore} windpark op zee gerealiseerd door Cleaneco.
\item Er wordt geen verantwoordelijkheid genomen voor problemen die zich tijdens de realisatiefase van het \gls{offshore} windpark op zee voordoen.
\end{itemize}
\subsection{Verwachte tijdsduur}
Het project \textit{Duurzame Energie} is gestart op dinsdag 29 augustus en wordt gepland om 17 weken in beslag te nemen. In week 19 staat de eindpresentatie gepland voor de opdrachtgever Worley, die zal plaatsvinden op het kantoor in Den Haag. De geschatte tijdsindeling gedurende dit project is ongeveer 16 x 2 contacturen en 16 x 6 uren die zullen worden besteed aan onderzoek en realisatie van de producten.\cite{studiewijzer}.

% \section{Projectgrenzen}
% \subsection{Wat wel}
% \begin{itemize}
%     \item Wij schrijven een plan van aanpak.
%     \item Wij doen onderzoek naar de technische aspecten en de energie opbrengsten.
%     \item Wjj doen onderzoek naar de verschillende windturbines, transformatoren, bekabeling en toebehoren.
%     \item Wij schrijven een onderhoudsplan voor 25 jaar waarbij we gebruik maken van levensduur en statistische gegevens.
%     \item Wij maken een parkontwerp waarbij rekening wordt gehouden met een optimale plaatsing van de windturbines.
%     \item Wij presenteren ons parkontwerp en onderhoudsplan aan onze opdrachtgever: Worley.
% \end{itemize}
% \subsection{Wat niet}
% \begin{itemize}
%     \item Wij vragen geen vergunning aan voor een \gls{offshore} windpark op zee.
%     \item Wij vragen geen subsidie aan bij de Nederlandse overheid.
%     \item Wij beheren geen elektrische infrastructuur (wordt gedaan door TenneT\cite{energieakkoord}).
%     \item Wij realiseren geen \gls{offshore} windpark op zee.
%     \item Wij zijn niet verantwoordelijk voor oplopende problemen tijdens de realisatie fase van het \gls{offshore} windpark op zee.
% \end{itemize}
% \subsection{Verwachte tijdsduur}
% Op dinsdag 29 augustus zijn we begonnen aan het project \textit{Duurzame Energie} en we plannen om hier 17 weken aan te besteden. In week 19 staat onze eindpresentatie gepland voor onze opdrachtgever Worley, die zal plaatsvinden op hun kantoor in Den Haag. 
% We schatten in dat we gedurende dit project ongeveer 16 x 2 contacturen en 16 x 6 zelfstudie-uren zullen besteden\cite{studiewijzer}.


\section{Samenvatting}
Voor dit project, Project Windpark Op Zee is het bedrijf Cleaneco in dienst genomen door Worley. 
Er moet een windpark ontworpen worden voor de kavels VI en VII van Hollandse Kust West. Er moet daarbij ook een onderhoudsplan voor de komende 25 jaar geleverd worden. Dit windpark moet een vermogen realiseren van 1500MW. 

Bij het bepalen van de indeling van het windpark moeten verschillende obstakels ontweken worden. Deze obstakels zijn ontweken met 100m. Daarnaast is van belang wat de windsnelheden en -richtingen op de kavels zijn. Dit is bepaald op basis van winddata van over een periode van twee decennia. Door deze data te filteren is de overheersende windrichting bepaald. Bij deze windrichting is een marge van 45$^{\circ}$ beide richtingen op, opgeteld om tot de windrichtingen te komen die voldoen aan de voorwaarde om mee door te kunnen rekenen. De windrichtingen die voldoen zitten tussen de 152,7$^{\circ}$ en 242,7$^{\circ}$. 
Er zijn vervolgens twee turbines in 2 configuraties gekozen welke vergeleken zijn op veel verschillende factoren. De eerste turbine is een 18MW turbine, de tweede een 12MW turbine. De 18MW turbine is de betere optie van de twee, met een jaarlijkse energieopbrengst van 3.571.586.112 kWh. Hierbij is ook al rekening gehouden met het parkeffect. Vanwege de grote rotoroppervlak is het maximaal aantal turbines per kavel gelimiteerd tot 48. Echter is deze turbine alsnog efficiënter en wekt het meer energie op dan 60 van de 12MW turbines. 
Om de indeling van het park te kunnen bepalen is de verwachte stroom belangrijk. Aan de hand hiervan kan worden bepaald hoeveel turbines er per tak (kabel) kunnen worden verbonden. Dit aantal komt uit op 4 turbines van 18MW. Hieruit is vervolgens het totale vermogen berekend, welke samen met de spanning gebruikt is om de werkelijke stroom te bereken. Deze is gelijk aan 629,84A. Hiervoor is de ABB XLPE drie core submarine kabel met een doorsnede van 800$mm^{2}$ genoeg om met een veilig marge te oppereren op het windpark. Door het elektrischmodel op te stellen van de kabel, zijn berekeningen mogelijk gemaakt waarmee de verliezen in de kabel bepaald zijn. Dit verschilt per kabel vanwege de variërende lengte. 

Het windpark moet ook onderhouden worden. Hiervoor is een onderhoudsplan opgesteld. Voordat dit kon gebeuren zijn eerst de componenten van het windpark en de windturbines uitgewerkt. Voor elk onderdeel is aangegeven wat deze zijn, wat hun functie is wat voor onderhoud genoodzaakt is voor het bewaken van de duurzaamheid. Hierbij is onderscheid gemaakt tussen groot onderhoud en klein onderhoud. Klein onderhoud is vaak routine onderhoud. Het bijvullen van koelvloeistoffen, een nieuwe coating toevoegen tegen corrosie en andere reparaties kleine maar toch essentiële onderdelen vallen onder klein onderhoud. Groot onderhoud omvat ingrijpende operaties waarbij componenten vaak volledig worden vervangen. In het onderhoudsplan is het voorkomen van falen ook een grote factor. Daarom zijn ook verschillende monitoring, preventieve en voorspellende systemen in het plan verwerkt. Hoe het onderhoud moet worden uitgevoerd is ook behandeld per onderdeel met de frequentie waarmee dit moet gebeuren. De frequentie hangt af van verschillende factoren, waaronder de levensduur van de componenten en de omstandigheden waarin deze zich bevinden. Inspecties zullen jaarlijks of halfjaarlijks plaatsvinden. Echter kan altijd per situatie bepaald worden wanneer onderhoud nodig is op basis van de monitoringssystemen. Dit zorgt voor waarborging van de duurzaamheid van alle componenten in het park. 

De manier waarop al deze rapporten opgesteld zullen worden is door middel van bronnenonderzoek. De bronnen zullen onder andere komen van de database van de Haagse Hogeschool en professionals in het betreffende vakgebied. Verder zullen betrouwbare bronnen zoals door de overheid verstrekte documenten gebruikt worden. Dit zal resulteren in een ontwerprapport van het gevraagde windturbinepark en het daar bijbehorende onderhoudsrapport voor de komende 25 jaar. 

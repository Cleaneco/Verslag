\section{Discussie}
\subsection{Turbines 360 Graden Laten Draaien}
Om de efficiëntie van het windpark te vergroten, overwegen we turbines volledig om hun as te laten draaien, ook wel bekend als free-floating turbines\cite{9136371}. Dit ontwerp kan zich beter aanpassen aan veranderende windrichtingen en -snelheden zonder de hele turbine te hoeven draaien, wat de energieopwekking optimaliseert. Implementatie van geavanceerde tracking- en positioneringstechnologie, zoals LiDAR (Light Detection and Ranging), is noodzakelijk om ervoor te zorgen dat de turbines altijd optimaal zijn uitgelijnd met de wind.\cite{8511422}

\subsection{Model Predictive Control (MPC) voor Turbine-Oriëntatie}
Een geavanceerde regelstrategie, Model Predictive Control (MPC), kan worden toegepast om de oriëntatie van de turbines te regelen. MPC gebruikt voorspellende modellen om optimale beslissingen te nemen, rekening houdend met variabelen zoals windvoorspellingen, energievraag en de staat van het turbinesysteem. Dit kan leiden tot een meer responsieve en efficiënte energieopwekking. Ontwikkeling van nauwkeurige voorspellende modellen en geavanceerde besturingssystemen die in realtime kunnen reageren op veranderende omstandigheden is essentieel.\cite{9255096}

\subsection{Overproductie in Waterstof Benutten}
Een innovatief idee is om overproductie om te zetten in waterstof. Waterstof kan worden opgeslagen en later worden gebruikt om elektriciteit te genereren tijdens periodes van verminderde windenergieproductie. Deze aanpak draagt bij aan een consistente en betrouwbare energielevering, zelfs tijdens onderhoudsperioden of bij lagere windintensiteit. Implementatie van geavanceerde elektrolysetechnologieën en slimme regelsystemen is noodzakelijk.\cite{10116157,9844498,8605079}

\subsection{Meer Onderzoek naar Impact op Ecosysteem/Milieu}
Het is van cruciaal belang om de impact van het windpark op zee op het ecosysteem en het milieu grondig te onderzoeken. Uitgebreide milieu-impactstudies, inclusief monitoring van zeevogels, zeezoogdieren en zeeleven, zijn noodzakelijk. Implementatie van beschermende maatregelen, zoals het gebruik van vogelradars van bedrijven zoals Robin Radar Systems\cite{robinradar-bird-detection-radar,robinradar-wind-farm-bird-radar} om vogelbotsingen te verminderen, is een waardevolle toevoeging. Het aanpassen van de locatie en het ontwerp van het windpark op basis van bevindingen uit milieuonderzoek is cruciaal om de ecologische voetafdruk zo minimaal mogelijk te houden.
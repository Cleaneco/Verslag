\section{Uit welke onderdelen bestaan het windpark en de windturbines en hoe word de conditie van deze componenten bewaakt?}
\subsection{Onderdelen windpark}
Een windpark bestaat uit meerdere hoofd onderdelen, welke zelf weer bestaan uit veel componenten. De zes hoofdonderdelen waaruit een windpark op zee bestaat zijn: 
\begin{itemize}
    \item De windturbines
    \item De fundaties
    \item De elektrische infrastructuur (inter array bekabeling)
    \item Substations op zee
    \item De export kabels
    \item Controle- en bewakingssystemen
\end{itemize}
Deze onderdelen bestaan elk weer uit componenten met hun eigen functies. 
De windturbines staan centraal in het windpark en zorgen ervoor dat de windenergie wordt omgezet tot elektrische energie.


\subsubsection{Windturbine componenten en systemen}
Een windturbine is een geavanceerd systeem dat bestaat uit verschillende cruciale componenten die elk met een specifieke functie om de efficiënte opwekking van elektriciteit uit windenergie mogelijk te maken.

\textbf{1. Rotorbladen:}
De rotorbladen vormen een essentieel onderdeel van de windturbine en zijn verantwoordelijk voor het vangen van de wind, wat de rest van de turbine in staat stelt de kinetische energie van de wind om te zetten. De rotorbladen zijn gemaakt van materialen zoals glasvezelversterkte kunststoffen, koolstofvezelversterkte kunststoffen of andere composietmaterialen, en zijn aerodynamisch ontworpen om de maximale hoeveelheid windenergie te vangen.\cite{Rotorbladen}\cite{Rotorbladen_windturbinecomponenten}

\textbf{2. Generator en Drive Train:}
De generator in een windturbine transformeert de mechanische energie die wordt gegenereerd door draaiende rotorbladen in elektriciteit. De drive train fungeert als het overdrachtsmechanisme dat deze energie van de rotor naar de generator transporteert. Twee veelvoorkomende configuraties zijn de traditionele tandwielkast en het direct drive-systeem. Beide configuraties hebben voor- en nadelen.

De tandwielkast verhoogt bij lage rotatiesnelheid van de rotor, de rotatie van de schacht, om de generator efficiënter te laten werken. Dit heeft als nadeel dat dit systeem meer onderhoud vergt, wat voor een windpark op zee extra uitdagingen met zich meebrengt.

Het direct drive-systeem daarentegen verbindt de rotor rechtstreeks met de generator. Dit zorgt ervoor dat bij lagere winden de generator minder efficiënt is. Het heeft echter ook als resultaat dat er minder onderdelen nodig zijn voor dit systeem, het veel betrouwbaarder is en het veel minder onderhoud vergt. De gekozen windturbines bevatten dan ook de DrivetrainSE module, een direct drive-systeem synchrone generator met permanente magneten.\cite{NREL_turbine_documentatie}

\textbf{4. Pompen, Koeling, Filters, Remmen:}
Belangrijke ondersteunende systemen omvatten pompen voor smering van alle mechanische componenten, koeling voor temperatuurregeling van kritieke componenten zoals de generator, filters voor het verwijderen van verontreinigingen uit de smeerolie, en remmen om schade bij te hoge windsnelheden en te snelle rotatie te voorkomen. Deze systemen dragen bij aan de levensduurverlenging van bewegende delen en verhogen de algehele betrouwbaarheid van de windturbine. Echter vergen ook deze onderdelen zelf onderhoud en bijvullen.

\textbf{5. Omvormers:}
Omvormers spelen een cruciale rol in het proces van elektriciteitsopwekking in de windturbine. Deze elektronische apparaten zetten de opgewekte wisselstroom (AC) van de generator om in gelijkstroom (DC) en vervolgens terug naar wisselstroom, waarbij de frequentie en spanning aangepast worden aan de specificaties van het elektriciteitsnet waarmee de turbine verbonden is. De synchronisatie met de frequentie van het elektriciteitsnet behoort ook tot de taken van de omvormer. Omvormers zijn dus cruciaal voor de netstabiliteit.

\textbf{6. Transformator(en):}
Transformators verhogen de spanning van de gegenereerde elektriciteit tot een niveau dat geschikt is voor transport over langere afstanden. In dit geval is dit 66 kV. Een efficiënte transformator is van essentieel belang voor de overdracht van elektriciteit over lange afstanden.

\textbf{7. Bladhoek (Pitch) Systeem:}
Het pitch-systeem regelt de hoek van de rotorbladen om de optimale invalshoek van de wind te behouden, waardoor de windturbine onder verschillende omstandigheden maximale energieopwekking kan realiseren.

\textbf{8. Krui (Yaw) Systeem:}
Het yaw-systeem past de oriëntatie van de gehele windturbine aan, waardoor de rotor voortdurend wordt gepositioneerd in de richting van de wind. Dit mechanisme waarborgt een continue efficiënte werking, ongeacht de variaties in windrichting.

\textbf{9. Nacelle:}
De nacelle is de behuizing bovenop de mast waarin zich de generator, omvormers en andere technische componenten bevinden. Het biedt bescherming aan deze essentiële systemen tegen weersinvloeden. 

\textbf{10. Mast:}
De mast is de paal die de windturbine boven het zeeniveau houdt. Gemaakt van duurzame materialen zoals staal of beton, biedt de mast de nodige hoogte om de rotorbladen bloot te stellen aan de krachtigste winden op grotere hoogtes. De hoogte van deze toren is 156m boven het gemiddelde zeeniveau. 

\textbf{12. Laagspannings Verdeelsysteem:}
Het laagspannings verdeelsysteem is een elektrisch systeem dat de opgewekte elektriciteit verdeelt naar verschillende componenten binnen de windturbine met behulp van laagspanning.

\textbf{13. Controlesysteem:}
Het controlemechanisme omvat sensoren, controllers en andere elektronische componenten die de prestaties van de windturbine monitoren en regelen. Het systeem zorgt ook voor de gegevensoverdracht naar de onshore Operations en Management locatie voor monitoring. Dit systeem is van cruciaal belang voor het optimaliseren van de energieopbrengst en het handhaven van operationele efficiëntie. Ook helpt dit systeem bij het voorkomen van falen door het faciliteren van informatie, op basis waarvan tijdig een beslissing kan worden genomen voor onderhoud. Het controlesysteem is dus een erg belangrijke onderdeel bij het bewaken van de duurzaamheid van de turbine.

\textbf{14. Utiliteiten:}
In de turbine zijn ook enkele utiliteiten zoals: 
\begin{itemize}
    \item HVAC (Heating, Ventilation, and Air Conditioning): Voor het regelen van de interne temperatuur van de nacelle.
    \item (Navigatie)verlichting: Om aan te geven aan vaartuigen en luchtvaartuigen waar de windturbine zich bevindt in geval van slecht zicht.
    \item Lift, Hijskranen: Faciliteiten voor onderhoudstoegang en het hijsen van zware apparatuur.
    \item Batterijsystemen: Opslag van elektriciteit voor noodstroomvoorziening.
    \item Brandblussyteem: Voor het bestrijden van branden en het beschermen van kritieke componenten.
\end{itemize}


In de keuze voor tandwielkast versus direct drive spelen factoren zoals locatie, schaal en budget een beslissende rol. Over het algemeen wordt direct drive vaak geprefereerd voor offshore windparken, waar de verminderde onderhoudsbehoefte en de uitdagingen van toegang tot de turbines doorslaggevende voordelen bieden.

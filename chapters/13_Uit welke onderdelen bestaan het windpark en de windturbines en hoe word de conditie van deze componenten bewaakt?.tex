\section{Uit welke onderdelen bestaan het windpark en de windturbines en hoe word de conditie van deze componenten bewaakt?}
\subsection{Onderdelen windpark}
Een windpark bestaat uit meerdere hoofd onderdelen, welke zelf weer bestaan uit veel componenten. De zes hoofdonderdelen waaruit een windpark op zee bestaat zijn\cite{Eneco_presentatie}: 
\begin{itemize}
    \item De windturbines
    \item De fundaties
    \item De elektrische infrastructuur (inter array bekabeling)
    \item Controle- en bewakingssystemen
\end{itemize}
Deze onderdelen bestaan elk weer uit componenten met hun eigen functies. 
De windturbines staan centraal in het windpark en zorgen ervoor dat de windenergie wordt omgezet tot elektrische energie. 
 
\subsubsection{Windturbine componenten en systemen}
Een windturbine is een geavanceerd systeem dat bestaat uit verschillende cruciale componenten die elk met een specifieke functie om de efficiënte opwekking van elektriciteit uit windenergie mogelijk te maken.

\textbf{1. Rotorbladen:}
De rotorbladen vormen een essentieel onderdeel van de windturbine en zijn verantwoordelijk voor het vangen van de wind, wat de rest van de turbine in staat stelt de kinetische energie van de wind om te zetten. De rotorbladen zijn gemaakt van materialen zoals glasvezelversterkte kunststoffen, koolstofvezelversterkte kunststoffen of andere composietmaterialen, en zijn aerodynamisch ontworpen om de maximale hoeveelheid windenergie te vangen.\cite{Rotorbladen}\cite{Rotorbladen_windturbinecomponenten}

\textbf{2. Generator en Drive Train:}
De generator in een windturbine transformeert de mechanische energie die wordt gegenereerd door draaiende rotorbladen in elektriciteit. De drive train fungeert als het overdrachtsmechanisme dat deze energie van de rotor naar de generator transporteert. Twee veelvoorkomende configuraties zijn de traditionele tandwielkast en het direct drive-systeem. Beide configuraties hebben voor- en nadelen.\cite{Huygens_Engineers_presentatie}

De tandwielkast verhoogt bij lage rotatiesnelheid van de rotor, de rotatie van de schacht, om de generator efficiënter te laten werken. Dit heeft als nadeel dat dit systeem meer onderhoud vergt, wat voor een windpark op zee extra uitdagingen met zich meebrengt.

Het direct drive-systeem daarentegen verbindt de rotor rechtstreeks met de generator. Dit zorgt ervoor dat bij lagere winden de generator minder efficiënt is. Het heeft echter ook als resultaat dat er minder onderdelen nodig zijn voor dit systeem, het veel betrouwbaarder is en het veel minder onderhoud vergt. De gekozen windturbines bevatten dan ook de DrivetrainSE module, een direct drive-systeem synchrone generator met permanente magneten.\cite{NREL_turbine_documentatie}

\textbf{4. Pompen, Koeling, Filters, Remmen:}
Belangrijke ondersteunende systemen omvatten pompen voor smering van alle mechanische componenten, koeling voor temperatuurregeling van kritieke componenten zoals de generator, filters voor het verwijderen van verontreinigingen uit de smeerolie, en remmen om schade bij te hoge windsnelheden en te snelle rotatie te voorkomen. Deze systemen dragen bij aan de levensduurverlenging van bewegende delen en verhogen de algehele betrouwbaarheid van de windturbine. Echter vergen ook deze onderdelen zelf onderhoud en bijvullen.\cite{Eneco_presentatie}

\textbf{5. Omvormers:}
Omvormers spelen een cruciale rol in het proces van elektriciteitsopwekking in de windturbine. Deze elektronische apparaten zetten de opgewekte wisselstroom (AC) van de generator om in gelijkstroom (DC) en vervolgens terug naar wisselstroom, waarbij de frequentie en spanning aangepast worden aan de specificaties van het elektriciteitsnet waarmee de turbine verbonden is. De synchronisatie met de frequentie van het elektriciteitsnet behoort ook tot de taken van de omvormer. Omvormers zijn dus cruciaal voor de netstabiliteit.

\textbf{6. Transformator(en):}
Transformators verhogen de spanning van de gegenereerde elektriciteit tot een niveau dat geschikt is voor transport over langere afstanden. In dit geval is dit 66 kV. Een efficiënte transformator is van essentieel belang voor de overdracht van elektriciteit over lange afstanden.\cite{Huygens_Engineers_presentatie}

\textbf{7. Bladhoek (Pitch) Systeem:}
Het pitch-systeem regelt de hoek van de rotorbladen om de optimale invalshoek van de wind te behouden, waardoor de windturbine onder verschillende omstandigheden maximale energieopwekking kan realiseren.\cite{NREL_turbine_documentatie}

\textbf{8. Krui (Yaw) Systeem:}
Het yaw-systeem past de oriëntatie van de gehele windturbine aan, waardoor de rotor voortdurend wordt gepositioneerd in de richting van de wind. Dit mechanisme waarborgt een continue efficiënte werking, ongeacht de variaties in windrichting.\cite{NREL_turbine_documentatie}

\textbf{9. Nacelle:}
De nacelle is de behuizing bovenop de mast waarin zich de generator, omvormers en andere technische componenten bevinden. Het biedt bescherming aan deze essentiële systemen tegen weersinvloeden. 

\textbf{10. Mast:}
De mast is de paal die de windturbine boven het zeeniveau houdt. Gemaakt van duurzame materialen zoals staal of beton, biedt de mast de nodige hoogte om de rotorbladen bloot te stellen aan de krachtigste winden op grotere hoogtes. De hoogte van deze toren is 156m boven het gemiddelde zeeniveau. 

\textbf{12. Laagspannings Verdeelsysteem:}
Het laagspannings verdeelsysteem is een elektrisch systeem dat de opgewekte elektriciteit verdeelt naar verschillende componenten binnen de windturbine met behulp van laagspanning.

\textbf{13. Utiliteiten:}
In de turbine zijn ook enkele utiliteiten zoals: 
\begin{itemize}
    \item HVAC (Heating, Ventilation, and Air Conditioning): Voor het regelen van de interne temperatuur van de nacelle.
    \item (Navigatie)verlichting: Om aan te geven aan vaartuigen en luchtvaartuigen waar de windturbine zich bevindt in geval van slecht zicht.
    \item Lift, Hijskranen: Faciliteiten voor onderhoudstoegang en het hijsen van zware apparatuur.
    \item Batterijsystemen: Opslag van elektriciteit voor noodstroomvoorziening.
    \item Brandblussyteem: Voor het bestrijden van branden en het beschermen van kritieke componenten.
\end{itemize}

\subsection{Fundaties}
Er zijn meerdere verschillende soorten fundaties. De gekozen fundatie voor de turbines is de monopile-fundatie. Monopiles zijn cilindrische stalen buizen van hoogwaardig staal en dienen als duurzame en kosteneffectieve funderingen voor offshore windturbines. Hun ontwerp, met zorgvuldig gekozen afmetingen en een taps toelopend ontwerp, maakt eenvoudige installatie mogelijk en biedt stabiliteit tegen zowel verticale als horizontale belasting. Het 'transition piece' boven op de monopile vormt de verbinding met de turbine, waardoor monopiles een betrouwbare keuze zijn.\cite{Monopile_Fundatie}

\subsection{De elektrische infrastructuur (inter array bekabeling)}
De inter array bekabeling vormt een cruciaal onderdeel van de elektrische infrastructuur in een windturbinepark. Het bestaat uit een netwerk van kabels dat de individuele windturbines in het park verbindt met het centrale verzamelpunt, het TenneT transformatorstation. Er is onderscheid te maken tussen inter array kabels en export kabels. De export kabels exporteren de door het windpark opgewekte elektriciteit van het transformatorstation naar het transformatorstation op land. De kabels in het windpark, die de turbines met elkaar (in groepen, arrays) en vervolgens met het offshore transformatorstation verbinden zijn de inter array kabels. Een belangrijke eigenschap van deze kabels is de hoogwaardige isolatie materialen, deze moeten de kabel beschermen tegen omgevingsfactoren zoals zout water en extreme weersomstandigheden. Daarnaast moeten de materialen corrosie bestendig zijn.

\subsection{Controle- en bewaking systemen}
Het controlemechanisme omvat sensoren, controllers en andere elektronische componenten die de prestaties van de windturbine monitoren en regelen. Het systeem zorgt ook voor de gegevensoverdracht naar de onshore Operations en Management locatie voor monitoring. Dit systeem is van cruciaal belang voor het optimaliseren van de energieopbrengst en het handhaven van operationele efficiëntie. Ook helpt dit systeem bij het voorkomen van falen door het faciliteren van informatie, op basis waarvan tijdig een beslissing kan worden genomen voor onderhoud. Het controlesysteem is dus een erg belangrijke onderdeel bij het bewaken van de duurzaamheid van de turbine.

\subsection{Kwaliteitsbewaking}

In het streven naar een uiterst effectief onderhoudsplan voor het windpark op Kavel VI en VII van Hollandse Kust West, worden geavanceerde technologieën en systemen ingezet. Een cruciaal aspect van dit plan is de implementatie van state-of-the-art sensoren op strategische locaties, die niet alleen de turbines zelf, maar ook de kabelknooppunten bij de turbines bewaken.

\subsubsection{Turbine Onderhoud}

Bij de turbines worden sensoren op vitale punten geplaatst, zoals:

\begin{itemize}
    \item \textbf{Turbinebehuizing:} Trillingsensoren en vibratiesensoren op de turbinebehuizing detecteren mechanische onregelmatigheden en structurele veranderingen. Siemens' Simcenter, geïntegreerd in het Digital Twin-platform, analyseert deze gegevens real-time en stelt het onderhoudsteam in staat om potentiële problemen vroegtijdig te identificeren\cite{5608150}.
    
    \item \textbf{Rotorbladen:} Ultrasoonsensoren bewaken de oppervlakte-integriteit van de rotorbladen en identificeren tekenen van scheurvorming. De verzamelde gegevens worden geanalyseerd in het Digital Twin-platform, waardoor het onderhoudsteam proactief maatregelen kan nemen om de levensduur van de rotorbladen te verlengen\cite{5608150}.
    
    \item \textbf{Nacelle:} Temperatuursensoren meten de temperatuur binnen de Nacelle, wat waardevolle informatie oplevert over de operationele omstandigheden. Door deze gegevens te integreren in het Digital Twin-platform, kan het onderhoudsteam anticiperen op temperatuurproblemen en preventieve maatregelen nemen\cite{5608150}.
    
    \item \textbf{Nacelle:} Stroomsensoren meten de elektrische stroom in de nacelle, terwijl ultrasoonsensoren interne defecten opsporen. Deze gegevens worden continu geanalyseerd, waardoor het onderhoudsteam een diepgaand inzicht krijgt in de elektrische en mechanische gezondheid van de turbine\cite{5608150}.
\end{itemize}

\subsubsection{Kabelknooppunt Onderhoud}

Voor de kabelknooppunten bij de turbines worden sensoren geplaatst op specifieke punten, zoals:

\begin{itemize}
    \item \textbf{Aansluitingen:} Temperatuursensoren, stroomsensoren en spanningssensoren bij de aansluitpunten van de kabels op de turbines identificeren problemen met de elektrische verbindingen. Deze gegevens worden geïntegreerd in het Digital Twin-platform, waardoor een holistisch beeld van het energietransmissiesysteem ontstaat\cite{siemens-digital-twin,siemens-x-digital-twin,siemens-product-10313567,siemens-product-10045207,siemens-power-quality-measurement}.

    \item \textbf{Kabelroutes:} Trillingsensoren langs de kabelroutes detecteren structurele veranderingen in de kabels, zoals slijtage of beschadiging. Het Digital Twin-platform analyseert deze gegevens en biedt inzicht in mogelijke onderhoudsbehoeften\cite{siemens-digit1al-twin,siemens-x-digital-twin,siemens-product-10313567,siemens-product-10045207,siemens-power-quality-measurement}.
    
    \item \textbf{Verbindingspunten:} Ultrasoonsensoren bij verbindingspunten in de kabelinfrastructuur bieden inzicht in mogelijke defecten in de isolatie. Door deze sensoren te integreren in het Digital Twin-platform, kan het onderhoudsteam proactief isolatieproblemen aanpakken\cite{siemens-digital-twin,siemens-x-digital-twin,siemens-product-10313567,siemens-product-10045207,siemens-power-quality-measurement}.
\end{itemize}

\subsubsection{Onderhoudsintervallen en Predictive Analytics}

Deze geavanceerde benaderingen, van sensoren tot Digital Twin-implementaties\cite{10175262,10269370}, vormen een solide basis voor een effectief onderhoudsplan. De continue gegevensstroom wordt niet alleen gebruikt voor real-time monitoring, maar ook voor voorspellende analyses en simulaties. Siemens' Simcenter en MindSphere stellen het onderhoudsteam in staat om potentiële problemen te identificeren en op te lossen voordat ze zich in de praktijk manifesteren.\cite{5256318}

Daarnaast zijn de onderhoudsintervallen gebaseerd op specifieke condities en weersomstandigheden. Uit historische gegevens blijkt dat de turbinebehuizing eens in de twee jaar een uitgebreide inspectie vereist, terwijl de rotorbladen elke drie jaar grondig worden geïnspecteerd. De Nacelle, met zijn complexe elektrische componenten, wordt jaarlijks onderworpen aan gedetailleerd onderzoek, en de nacelle krijgt elke vier jaar een uitgebreide check-up.\cite{5460911}

Voor de kabelknooppunten gelden vergelijkbare schema's. Aansluitingen worden halfjaarlijks geïnspecteerd, terwijl kabelroutes jaarlijks worden gecontroleerd op structurele veranderingen. Verbindingspunten in de kabelinfrastructuur krijgen elke drie jaar een grondige inspectie. Deze preventieve onderhoudsintervallen worden ondersteund door de continue monitoring van de sensoren en de voorspellende analyses van het Digital Twin-platform, wat resulteert in een nauwkeurig afgestemd en kostenefficiënt onderhoudsprogramma.

De voordelen van dit geavanceerde onderhoudsplan zijn tweeledig: het waarborgt niet alleen de betrouwbaarheid van het windpark door vroegtijdige detectie van mogelijke storingen, maar optimaliseert ook de operationele efficiëntie en minimaliseert ongeplande stilstand, wat resulteert in aanzienlijke kostenbesparingen en een verbeterde duurzaamheid van het gehele systeem\cite{5460911,5256318,10269370}.

\subsubsection{Power Quality Monitoring}
Een aanvullend aspect van het onderhoudsplan is het monitoren van de power quality in het windpark. Siemens' SICAM Power Quality Measurement wordt ingezet om de stabiliteit van het elektriciteitsnet te waarborgen. Veel elektronische apparaten en automatiseringssystemen in industriële productie-installaties zijn gevoelig voor spanningsvariaties en dips in de stroomtoevoer, vaak veroorzaakt door onverwachte aansluiting van energiebronnen zoals hernieuwbare energie en grote energieverbruikers\cite{siemens-power-quality-measurement}. Deze schommelingen kunnen productiestoringen en verlies veroorzaken, wat het belang benadrukt van continue meting en analyse van de power quality. Het SICAM-systeem biedt de nodige gegevens om onverwachte risico's vroegtijdig te detecteren en proactief te handelen\cite{siemens-power-quality-measurement}.

\subsubsection{Monopaal Onderhoud}
Voor het onderhoud van de monopalen worden geavanceerde sensoren ingezet op cruciale punten, waaronder:
\begin{itemize}
    \item \textbf{Structurele Integriteit:} Trillingsensoren en ultrasoonsensoren op de monopaal bewaken de structurele integriteit, detecteren mechanische onregelmatigheden en identificeren tekenen van slijtage of beschadiging. Deze gegevens worden geanalyseerd in het Digital Twin-platform, waardoor het onderhoudsteam proactieve maatregelen kan nemen om de levensduur van de monopalen te optimaliseren \cite{5406794}.
    \item \textbf{Corrosie:} Corrosiesensoren meten corrosieniveaus op de monopalen. Door deze gegevens te integreren in het Digital Twin-platform, kan het onderhoudsteam de corrosiestatus monitoren en tijdig ingrijpen om verdere schade te voorkomen \cite{5406794}.
    \item \textbf{Omgevingsfactoren:} Temperatuur- en weerssensoren bieden waardevolle informatie over de omgevingsfactoren rondom de monopalen. Deze gegevens helpen bij het anticiperen op extreme omstandigheden en het plannen van preventief onderhoud \cite{5406794}.
\end{itemize}

De implementatie van deze sensoren, gekoppeld aan het Digital Twin-platform, vormt een integraal onderdeel van het uitgebreide onderhoudsplan, wat resulteert in een betrouwbare werking en verlengde levensduur van de monopalen op het windpark van Kavel VI en VII van Hollandse Kust West. \cite{WPE}


\subsubsection{Milieu- en Externe Risicobewaking}
Naast het monitoren van de structurele integriteit van de monopalen, is het onderhoudsplan uitgerust met sensoren die de invloed van het milieu en externe risico's op het windpark evalueren. Dit omvat:

\begin{itemize}
    \item \textbf{Weercondities:} Weersensoren meten parameters zoals windsnelheid, luchtvochtigheid en temperatuur. Deze gegevens bieden inzicht in de omgevingscondities rond de monopalen. Bij extreme weersomstandigheden kan het Digital Twin-platform waarschuwingen genereren, waardoor het onderhoudsteam preventieve maatregelen kan nemen om mogelijke schade te voorkomen \cite{9738001}.
    \item \textbf{Zoutwatercorrosie:} Gezien de offshore-locatie van het windpark worden corrosiesensoren specifiek toegepast om zoutwatercorrosie op de monopalen te monitoren. Deze sensoren meten de corrosiegraad en waarschuwen het onderhoudsteam voor potentieel schadelijke niveaus, waardoor tijdig ingrijpen mogelijk is \cite{9738001}.
    \item \textbf{Scheepvaartrisico's:} Om de kans op aanvaringen met schepen te minimaliseren, worden sensoren geïntegreerd die de nabijheid van schepen rondom de monopalen detecteren. Dit systeem kan het onderhoudsteam tijdig informeren over naderende schepen, waardoor maatregelen genomen kunnen worden om botsingen te voorkomen \cite{9738001}.
\end{itemize}

Deze uitgebreide monitoring van milieu- en externe risicofactoren draagt bij aan een proactief onderhoudsbeleid, waarbij niet alleen de fysieke integriteit van de monopalen wordt bewaakt, maar ook potentiële bedreigingen vanuit de omgeving worden geïdentificeerd en geminimaliseerd. Dit resulteert in een duurzaam en robuust windparkbeheer op Kavel VI en VII van Hollandse Kust West.

\subsection{Kwaliteitsbewaking}

In het streven naar een uiterst effectief onderhoudsplan voor het windpark op Kavel VI en VII van Hollandse Kust West, worden geavanceerde technologieën en systemen ingezet. Een cruciaal aspect van dit plan is de implementatie van state-of-the-art sensoren op strategische locaties, die niet alleen de turbines zelf, maar ook de kabelknooppunten bij de turbines bewaken.

\subsubsection{Turbine Onderhoud}

Bij de turbines worden sensoren op vitale punten geplaatst, zoals:

\begin{itemize}
    \item \textbf{Turbinebehuizing:} Trillingsensoren en vibratiesensoren op de turbinebehuizing detecteren mechanische onregelmatigheden en structurele veranderingen. Siemens' Simcenter, geïntegreerd in het Digital Twin-platform, analyseert deze gegevens real-time en stelt het onderhoudsteam in staat om potentiële problemen vroegtijdig te identificeren\cite{5608150}.
    
    \item \textbf{Rotorbladen:} Ultrasoonsensoren bewaken de oppervlakte-integriteit van de rotorbladen en identificeren tekenen van scheurvorming. De verzamelde gegevens worden geanalyseerd in het Digital Twin-platform, waardoor het onderhoudsteam proactief maatregelen kan nemen om de levensduur van de rotorbladen te verlengen\cite{5608150}.
    
    \item \textbf{Nacelle:} Temperatuursensoren meten de temperatuur binnen de Nacelle, wat waardevolle informatie oplevert over de operationele omstandigheden. Door deze gegevens te integreren in het Digital Twin-platform, kan het onderhoudsteam anticiperen op temperatuurproblemen en preventieve maatregelen nemen\cite{5608150}.
    
    \item \textbf{Nacelle:} Stroomsensoren meten de elektrische stroom in de nacelle, terwijl ultrasoonsensoren interne defecten opsporen. Deze gegevens worden continu geanalyseerd, waardoor het onderhoudsteam een diepgaand inzicht krijgt in de elektrische en mechanische gezondheid van de turbine\cite{5608150}.
\end{itemize}

\subsubsection{Kabelknooppunt Onderhoud}

Voor de kabelknooppunten bij de turbines worden sensoren geplaatst op specifieke punten, zoals:

\begin{itemize}
    \item \textbf{Aansluitingen:} Temperatuursensoren, stroomsensoren en spanningssensoren bij de aansluitpunten van de kabels op de turbines identificeren problemen met de elektrische verbindingen. Deze gegevens worden geïntegreerd in het Digital Twin-platform, waardoor een holistisch beeld van het energietransmissiesysteem ontstaat\cite{siemens-digital-twin,siemens-x-digital-twin,siemens-product-10313567,siemens-product-10045207,siemens-power-quality-measurement}.

    \item \textbf{Kabelroutes:} Trillingsensoren langs de kabelroutes detecteren structurele veranderingen in de kabels, zoals slijtage of beschadiging. Het Digital Twin-platform analyseert deze gegevens en biedt inzicht in mogelijke onderhoudsbehoeften\cite{siemens-digit1al-twin,siemens-x-digital-twin,siemens-product-10313567,siemens-product-10045207,siemens-power-quality-measurement}.
    
    \item \textbf{Verbindingspunten:} Ultrasoonsensoren bij verbindingspunten in de kabelinfrastructuur bieden inzicht in mogelijke defecten in de isolatie. Door deze sensoren te integreren in het Digital Twin-platform, kan het onderhoudsteam proactief isolatieproblemen aanpakken\cite{siemens-digital-twin,siemens-x-digital-twin,siemens-product-10313567,siemens-product-10045207,siemens-power-quality-measurement}.
\end{itemize}

\subsubsection{Onderhoudsintervallen en Predictive Analytics}

Deze geavanceerde benaderingen, van sensoren tot Digital Twin-implementaties\cite{10175262,10269370}, vormen een solide basis voor een effectief onderhoudsplan. De continue gegevensstroom wordt niet alleen gebruikt voor real-time monitoring, maar ook voor voorspellende analyses en simulaties. Siemens' Simcenter en MindSphere stellen het onderhoudsteam in staat om potentiële problemen te identificeren en op te lossen voordat ze zich in de praktijk manifesteren.\cite{5256318}

Daarnaast zijn de onderhoudsintervallen gebaseerd op specifieke condities en weersomstandigheden. Uit historische gegevens blijkt dat de turbinebehuizing eens in de twee jaar een uitgebreide inspectie vereist, terwijl de rotorbladen elke drie jaar grondig worden geïnspecteerd. De Nacelle, met zijn complexe elektrische componenten, wordt jaarlijks onderworpen aan gedetailleerd onderzoek, en de nacelle krijgt elke vier jaar een uitgebreide check-up.\cite{5460911}

Voor de kabelknooppunten gelden vergelijkbare schema's. Aansluitingen worden halfjaarlijks geïnspecteerd, terwijl kabelroutes jaarlijks worden gecontroleerd op structurele veranderingen. Verbindingspunten in de kabelinfrastructuur krijgen elke drie jaar een grondige inspectie. Deze preventieve onderhoudsintervallen worden ondersteund door de continue monitoring van de sensoren en de voorspellende analyses van het Digital Twin-platform, wat resulteert in een nauwkeurig afgestemd en kostenefficiënt onderhoudsprogramma.

De voordelen van dit geavanceerde onderhoudsplan zijn tweeledig: het waarborgt niet alleen de betrouwbaarheid van het windpark door vroegtijdige detectie van mogelijke storingen, maar optimaliseert ook de operationele efficiëntie en minimaliseert ongeplande stilstand, wat resulteert in aanzienlijke kostenbesparingen en een verbeterde duurzaamheid van het gehele systeem\cite{5460911,5256318,10269370}.

\subsubsection{Power Quality Monitoring}
Een aanvullend aspect van het onderhoudsplan is het monitoren van de power quality in het windpark. Siemens' SICAM Power Quality Measurement wordt ingezet om de stabiliteit van het elektriciteitsnet te waarborgen. Veel elektronische apparaten en automatiseringssystemen in industriële productie-installaties zijn gevoelig voor spanningsvariaties en dips in de stroomtoevoer, vaak veroorzaakt door onverwachte aansluiting van energiebronnen zoals hernieuwbare energie en grote energieverbruikers\cite{siemens-power-quality-measurement}. Deze schommelingen kunnen productiestoringen en verlies veroorzaken, wat het belang benadrukt van continue meting en analyse van de power quality. Het SICAM-systeem biedt de nodige gegevens om onverwachte risico's vroegtijdig te detecteren en proactief te handelen\cite{siemens-power-quality-measurement}.